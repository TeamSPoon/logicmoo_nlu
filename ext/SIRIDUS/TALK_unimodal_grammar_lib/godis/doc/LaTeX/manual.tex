\documentclass{book}


%\usepackage{robinmacros2}
%\usepackage{slmacros}
%\usepackage{frames}
\usepackage{mynamed}
%% Robin's macros
%
%

\newcommand{\ignore}[1]{}

%
% Reference the next example in the text:
%
\newcommand{\nexteg}[1]{\addtocounter{examplectr}{1}(\arabic{chapter}.\arabic{examplectr}{#1})\addtocounter{examplectr}{-1}}
%
% Reference the previous example in the text:
%
\newcommand{\preveg}[1]{(\arabic{chapter}.\arabic{examplectr}{#1})}
%
% Reference examples by relative offsets:
%
\newcommand{\egnum}[2]{\addtocounter{examplectr}{#1}(\arabic{chapter}.\arabic{examplectr}{#2})\addtocounter{examplectr}{-#1}}


%
% Examples Environments (RC)
%
\newcounter{examplectr}
\newcounter{subexamplectr}[examplectr]
\renewcommand{\thesubexamplectr}{\alph{subexamplectr}}
%
% Sub-example Macro:
%
\newenvironment{subex}%
   {\begin{list}
       {\alph{subexamplectr}.}%
       {\setlength{\topsep}{0in}
\setlength{\labelwidth}{1.0in}
        \setlength{\leftmargin}{0.25in}
        \setlength{\labelsep}{0.15in}
 \usecounter{subexamplectr}}
       }%
   {\end{list}}

%
% Single Example Macro
%
\newenvironment{singex}
{ \refstepcounter{examplectr}
  \bigskip % vertical
  \begin{list}
    {
      (\arabic{chapter}.\arabic{examplectr})\\
    }%
    {\setlength{\topsep}{0in}
\setlength{\labelwidth}{0.5in}
      \setlength{\leftmargin}{0.90in}
      \setlength{\labelsep}{0.10in}% distance counter - minipage
    }
    \begin{minipage}[t]{4.5in} \item
    }%
    {\end{minipage}
  \end{list}
}

\newcommand{\subegref}[2]{(\ref{#1}\ref{#2})}

\newcommand{\beg}{\begin{singex}}
\newcommand{\eeg}{\end{singex}}
\newcommand{\bseg}{\begin{subex}}
\newcommand{\eseg}{\end{subex}}


\newcommand{\egstack}[1]{\begin{tabular}[t]{@{}l@{}} #1 \\ \\ \end{tabular}}
% End of example macro

% trindikit module
\newcommand{\module}[1]{\textbf{#1}}

% TrindiKit ``logo''
\newcommand{\trindikit}[0]{\textsc{TrindiKit}}



% GoDiS plans (SL)
\newcommand{\plan}[2]{\textsc{issue : }#1\\\textsc{plan}:#2}
\newcommand{\qplan}[2]{\textsc{issue : }#1\\\textsc{plan}:#2}
\newcommand{\aplan}[3]{\textsc{action : }#1\\\textsc{plan}:#2\\\textsc{post : }#3}
\newcommand{\sequence}[1]{$\langle$\\\vlist{#1}\\$\rangle$}
\newcommand{\shortsequence}[1]{$\langle$#1$\rangle$}

\newcommand{\algorithm}[1]{#1}

\newcommand{\sort}[1]{\textsc{#1}}
\newcommand{\sortsubcat}[2]{\sort{#1}$\left\{\vlist{#2}\right.$}

% type
\newcommand{\type}[1]{#1}
% objects
\newcommand{\object}[1]{\textsf{#1}}
\newcommand{\objects}[1]{\textsf{\oplist{#1}}}
% actions are objects
\newcommand{\action}[1]{\object{#1}}


% (rule) class
\newcommand{\class}[1]{\object{#1}}

% content
%\newcommand{\content}[1]{#1}
\newcommand{\cont}[1]{\textnormal{\textbf{#1}}}
\newcommand{\content}[1]{\cont{#1}}

% content type (question, proposition etc)
\newcommand{\conttype}[1]{\type{#1}}


% path
\newcommand{\path}[1]{\textsc{#1}}
\newcommand{\sel}[1]{\textsc{#1}}

% path separator (outside implementation)
\newcommand{\s}[1]{/}


% TIS variable
\newcommand{\tisvar}[1]{\path{#1}}

% type declaration (variable is of type)
\newcommand{\oftype}[0]{\textbf{:}\/}
\newcommand{\typedecl}[2]{#1 \oftype\/ \type{#2}
}

% rule name
\newcommand{\rulename}[1]{\textbf{#1}}
%  update rule
\newcommand{\urule}[4]{
    {\sc rule:} \rulename{#1}\\
    \textsc{class:} \class{#2}\\
    {\sc pre:}  \oplist{
      \op{#3}
      }\\
    {\sc eff:} \oplist{
      \op{#4}
      }
  }

\newcommand{\concept}[1]{\textsf{#1}}

% dialogue move
\newcommand{\move}[1]{\object{#1}}
% \newcommand{\move}[2]{\class{#1}(\content{#2})}

%Records

\newcommand{\record}[1]{$\left[\mbox{\begin{tabular}{lcl}#1\end{tabular}}\right]$} 
\newcommand{\field}[2]{\path{#1}&=&#2}
\newcommand{\tfield}[2]{\path{#1} & : & \type{#2}}

\newcommand{\vlist}[1]{\begin{tabular}{l}#1\end{tabular}}
 
% set

\newcommand{\set}[1]{$\left\{\vlist{#1}\right\}$}

% stack

\newcommand{\stack}[1]{$\langle\vlist{#1}\rangle$}

% big []

\newcommand{\sqbrack}[1]{$\left[\vlist{#1}\right]$}

% big ()

\newcommand{\roundbrack}[1]{$\left(\vlist{#1}\right)$}





\newcommand{\prolog}[1]{\texttt{#1}}
\newcommand{\godis}[0]{\textsc{GoDiS}}
%\newcommand{\ibis}[0]{\godis}
\bibliographystyle{mynamed}

\parindent = 0pt
\setlength{\parskip}{\baselineskip}

\title{Manual for GoDiS\\DRAFT}
\author{Staffan Larsson}


\begin{document}

\maketitle

\tableofcontents

This manual is a work in progress.


\chapter{Introduction}



This document is intended as a manual for depeloping applications for
GoDiS, an Issue-based dialogue system. For an explanation of the
issue-based theory and its implementation in GoDiS, see \cite{larsson:thesis}.




\chapter{Getting started}

First, you need to download and install TrindiKit and GoDiS. See
Appendix \ref{cha:install} for instructions. Optionally, you may also
want to install OAA and any speech recognition and/or TTS engine that
is installed on your computer\footnote{Note that you need to rebuild
  your TrindiKit installation after you have installed OAA, Nuance
  v8.0 or Nuance Vocalizer.}. It is also very useful to have a text
editor such as Emacs installed.






\section{Running an existing GoDiS application}


To run the GoDiS VCR application using text input and output, open the
file \prolog{start-vcr-text.pl} located in you GoDiS directory under
\texttt{godis-apps/domain\_vcr} and consult it. When the prolog prompt
reappears, type ``\texttt{run.}'' and press return. When the user
input prompt (\texttt{\$U$>$}) appears, you may type e.g.
``\texttt{add a program}'' (without quotes). Top the dialogue e.g. by
typing ``\texttt{bye}''. If you don't want to see the rules and
information states, stop the dialogue and type ``\texttt{quiet.}''. To
see the rules again, type ``\texttt{verb.}'' (for ``verbose mode).


\section{Copying and modifying an existing GoDiS application}



A good way to get started on your own application is to copy an
existing application directory and successively replacing the
application components.

\ignore{
\section{Running the GUI}

To run GoDiS with the \trindikit GUI (Graphical User Interface), do
the following after loading but before running \godis{}:

\texttt{setflag(gui,server)}

Then, run the system as usual. This will start several DOS (or xterm)
windows, one of which is called "tis". In this window, a port number
will eventually appear. Now, start the GUI:

\texttt{java -cp \$TRINDIKIT/classes se.gu.ling.trindikit.gui.TrindiGui}

In the GUI, select File $\rightarrow$ Open Session. You are requested
to give a host and a port number; the latter should be the one
displayed in the "tis" window.

Under Windows, for \texttt{host} type \texttt{localhost} or \texttt{127.0.0.1}.
}




\chapter{Elements of the information-state approach to dialogue management }

In this section, we briefly outline the information state approach to
dialogue management. We also explain the relation between toolkits,
dialogue systems, and dialogue system applications. For a more
in-depth presentation of the information state apporach, see
\cite{larsson_traum:nle} or \cite{traum&larsson:isadm}.


\section{The information state approach}

The basic idea behind the ``information state approach'' is fairly
simple. To begin with, we regard dialogue as a kind of game, where
certain \textit{dialogue moves} are possible. A common kind of
dialogue move is a verbal utterance. Each move in a dialogue has
effects on some kind of state containing information, and each new
move is selected based on such a state.  This is the basic idea, and
it can be made more concrete, e.g., by considering the information
state as (a part of) a \emph{mental} state of some agent.

Dialogue is not, however, merely talking; it is also
\textit{thinking}. Thinking can also, at least to some extent, be
regarded as successive updates to an information state. To model
thinking in this way, we use information state \textit{update rules}
which have the form of conditionals: if $x$ holds of the current
state, then modify the state by applying operation $y$. For example,
one could implement a rule saying that ``if $p$ is in the current
state, and $p\rightarrow q$ is also in the current state, then add $q$
to the state''. This amounts to a forward-chanining modus ponens rule.
In keeping with the ``dialogue as game'' terminology, we can regard
update rules as ``silent moves''.

We make one cruicial, and perhaps not obvious, assumption about the
relation between dialogue moves and update rules. \textit{The relation
  between dialogue moves and information states can be captured
  completely by update rules.} That is, we don't have any
representation of the preconditions and effects of dialogue moves
beyond what is given in the update rules. The effects of a dialogue
move $M$ will typically be represented in a rule which has as a
precondition that $M$ was performed recently, and has not yet been
integrated.

Given what we have said so far, the information state is essentially a
``blackboard'' structure [REF to hearsay II etc]. However, we assume
that the information state is not just an unstructured jumble of
information; it is an \texttt{object} of a certain \texttt{datatype}.
In object-oriented programming, one defines classes of objects, and
for each class a set of methods. A datatype is similar to a class;
however, when definining datatypes we make a distinction between
operations (which modify objects of that type) and relations,
functions and selectors (which do not modify objects).

To sum up, the information state update approach to dialogue
management views utterances as dialogue moves which update, and are
selected on the basis of, a structured information state by means of
update rules. As such, this approach is fairly general and allows the
implementation of many different theories of dialogue. 


\section{\trindikit{} and the information state approach}



How does \trindikit{} relate to all this? Actually, \trindikit{}
currently makes some assumptions not inherent in the information state
update appoach. For one thing, it requires that modules in a dialogue
system do not communicate directly with one another, but \textit{only
  via the information state}. The purpose of this is to some extent
ideological; we belileve it is generally a good idea to have all the
information processed in the system visible, both to other modules and
to the designer of the system. If we force all communication to go via
the information state, we guarantee that by looking and the successive
updates to the information state we can completely capture all the
interactions between the modules of our system. There are also other
advantages; for example, keeping all information in the IS means that
the modules don't have to have any information about each other; all
they need to know is ehere to read and where to write in the IS. This,
in effect, means that one module can be exchanged for another without
causing disturbances or requiring modifications in other modules.
Unfortunately, this strategy of keeping all information in the IS has
one major disadvantage: the IS tends to become a bottleneck, slowing
down the system. There are various ways of improving the situation,
which we will not go into here; suffice to say that we are considering
lifting this limitation in future releases of \trindikit{}.

An additional requirement posed by \trindikit{} is that the algorithms
controlling the internal distribution of work in the system (i.e.,
when each module should start and stop working) is separate from
algorithms for updating the information state and selecting dialogue
moves. This simply makes sense to us as a design principle, and again
the argument is modularity. Doing things this way allows us to modify
the control algorithm of a dialogue system independently of any other
components.

Are there any theories which cannot be implemented using this
approach? Well, a basic requirement is of course that the theory is or
can be formalized


\section{Toolkits, dialogue systems and applications}




\section{Genre-specific systems: \godis-IOD and \godis-AOD}





\chapter{Elements of Issue-based dialogue management}

This section is intended to give a brief overview of the\godis{} system
and how it works.




\section{Total Information State}



\subsection{Information state proper}

The Information State (IS) is the main component of the Total
Information State (TIS). 

There is a basic division of IS into a record of information private
to the system, and a record representing shared information.


\begin{figure*}[htbp]
\centering
  \record{ 
    \tfield{private}{ 
      \record{
        \tfield{agenda}{OpenQueue(Action)}\\
        \tfield{plan}{OpenStack(PlanConstruct)}\\
        \tfield{bel}{Set(Prop)}\\
        \tfield{tmp}{
          \record{
            \tfield{usr}{
              $Tmp$
            }\\
            \tfield{sys}{
              $Tmp$
            }\\
          }
        }\\
        \tfield{nim}{OpenQueue(Pair(DP, Move))}\\
      }
    }\\
    \tfield{shared}{ 
      \record{
        \tfield{com}{Set(Prop)}\\
        \tfield{issues}{OpenStack(Question)}\\
        \tfield{actions}{OpenStack(Action)}\\
        \tfield{qud}{OpenStack(Question)}\\
        \tfield{pm}{OpenQueue(Move)}\\
        \tfield{lu}{
          \record{
            \tfield{speaker}{Participant}\\
            \tfield{moves}{Set(Move)}\\
                                %            \tfield{imoves}{Set(Move)} 
          }  
        }\\
      }
    }\\
  }\\


$Tmp$=
  \record{
    \tfield{com}{Set(Prop)}\\
    \tfield{issues}{OpenStack(Question)}\\
    \tfield{actions}{OpenStack(Action)}\\
    \tfield{qud}{OpenStack(Question)}\\
    \tfield{agenda}{OpenQueue(Action)}\\
    \tfield{plan}{OpenStack(PlanConstruct)}
  }

  
  \caption{\godis{-AOD} Information State type}
  \label{fig:ibis4-is}
\end{figure*}




\paragraph{Agenda}



The field \path{/private/agenda} is of type \type{Stack(Action)}. In
general, we try to use datastructures which are as simple as possible;
a stack is the simplest ordered structure so it is used as a default
datastructure where order is needed as long as it is sufficient for
the purposes at hand. The agenda is read by the selection rules to
determine the next dialogue move to be performed by the system.


\paragraph{Plan}



The \path{/private/plan} is a stack of plan constructs. Some of the
update rules for managing the plan have the form of rewrite rules
which process complex plan constructs until some action is topmost on
the plan. Other rules execute this action in case it is a system action
or move it to the agenda in case it is a move-related action.


\paragraph{Private beliefs}



In \godis{}, the field \path{/private/bel}, a set of propositions, is used
to store the results of database searches. Of course, the database
(and the domain knowledge, and the lexicon) can be seen as a part of
the system's private belief set, but in \path{/private/bel} we choose
to represent only propositions which are directly relevant to the task
at hand and which are the result of database searches. This is similar
to seeing the database as a set of implicit beliefs, and database
consultation as an inference process where implicit beliefs are made
explicit. The reason for using a set is that a set is the simplest
unordered datastructure.


\paragraph{Questions Under Discussion}

In \godis{} we define Questions Under Discussion, or QUD, to be an
open stack of questions that can be addressed using short answers. The
open stack has some set-like properties, but also retains a stack
structure in case it should be useful for ellipsis
resolution\footnote{Note that this is different from the way Ginzburg
  \cite{ginzburg} defines QUD\index{QUD}, i.e.  as containing
  questions which have been raised but not yet resolved, and thus
  currently under discussion. For reasons given in chapter 5 of
  \cite{larsson:thesis}, we have divided Ginzburgs QUD into two
  structures: QUD and Issues.}.


\paragraph{Issues}

The field \path{issues} contains all questions which have been raised
in a dialogue (explicitly or implicitly) but not yet resolved. It thus
contains a collection of current, or ``live'' issues. A suitable data
structure appears to be an open stack, i.e., a stack where non-topmost
elements can be accessed. This allows a non-rigid modelling of current
issues and task-related dialogue structure.




\paragraph{Actions}


The only addition from IOD to AOD is the \path{actions} field which
has been added to \path{/shared} and \path{/private/tmp}. We assume
the actions stack is an open stack, which is the same structure that
we use for \path{issues}.


\paragraph{Shared Committments}



The field \path{/shared/com} contains the set of propositions that the
user and the system have mutually agreed to during the dialogue. They
need not actually be believed by either participant; the important
thing is that the DPs have committed to these propositions, even if
only for the purposes of the conversation.

To reflect that the contents need not be true, or even privately
believed by the DPS, and because we are not using situation
semantics (where there is a distinction between facts
and propositions) we use the label ``commitments'' or ``committed
propositions'', abbreviated as \path{com}, instead of FACTS. These,
then, are propositions to which the DPs are (taken to be) jointly
committed.



\paragraph{Latest utterance}



In \path{/shared/lu} we represent information about the latest
utterance: the speaker, and the move realized by the utterance. We
assume for the moment that each utterance can realize only one
move. This assumption will be removed in the next chapter.

TMP

\paragraph{Temporary store}


To enable the system to backtrack if an optimistic assumption turns
out to be mistaken, relevant parts of the information state is kept in
\path{/private/tmp}. The \path{qud} and \path{com} fields may change
when integrating an \move{ask} or \move{answer} move, respectively.
The plan may also be modified, e.g., if a \action{raise} action is
selected. Finally, if any actions are on the agenda when selection
starts (which means they were put there during by the update module),
these may have been removed during the move selection process. 


\paragraph{Non-integrated moves}

Since several moves can be performed per turn, \godis{} needs some way of
keeping track of which moves have been interpreted. This is done by
putting all moves in \tisvar{latest\_moves} in a queue structure
called \path{nim}, for Non-Integrated Moves. This structure is
private, since it is an internal matter for the system how many moves
have been integrated so far. Once a move is assumed to be grounded on
the understanding level the move is added to the
\path{/shared/lu/moves} set. Since the move has now been understood on
the pragmatic level, the content of the move will be a question or a
full proposition (for short answers, the proposition resulting from
combining it with a question on QUD).

\paragraph{Previous moves}

To be able to detect irrelevant followups, \godis{} needs to know what
moves were performed (and grounded) in the previous utterance. These
are stored in the \path{/shared/pm} field.


\subsection{Module interface variables}


\subsection{Resource interface variables}




\section{Dialogue moves}

While dialogue move types are often defined in terms of sentence mood,
speaker intentions, and/or discourse relations (see e.g.
\cite{core97coding}\/), we opt for a different solution. In our
approach, the type of move realized by an utterance is determined by
the relation between the content of the utterance, and the activity in
which the utterance occurs.


\subsection{Core dialogue moves in \godis-IOD}


The following dialogue moves are used in \godis{}:

\begin{itemize}
\item \move{ask($q$)}, where $q$ \oftype\/ \conttype {Question}
\item \move{answer($a$)}, where $a$ \oftype\/ \conttype{ShortAns} or $a$ \oftype\/  \conttype{Proposition}
\item \move{greet} 
\item \move{quit}
\end{itemize}

In inquiry-oriented dialogue\index{inquiry-oriented dialogue}, the
central dialogue moves concern raising and addressing issues. This is
done by the \move{ask} and \move{answer} moves, respectively.
\ignore{; typically, the system will ask a number of questions that
  the user must answer in order for the system to be able to provide
  the desired service to the user.}  The \move{greet} and \move{quit}
moves are used in the beginning and end of dialogues to greet the user
and indicate that the dialogue is over, respectively.



\subsection{Grounding moves}




\subsection{Additional moves in \godis-AOD}


In addition to the dialogue moves listed above, \godis{-AOD} uses the following two moves:

\begin{itemize}
\item \move{request($\alpha$)}, where $\alpha$ \oftype\/ \conttype{Action}
\item \move{confirm($\alpha$)}, where  $\alpha$ \oftype\/  \conttype{Action}
\end{itemize}


These two moves are sufficient for activities where actions are
performed instantly or near-instantly, and always succeed. If these
requirements are not fulfilled, the \move{confirm} move can be
replaced by or complemented with a more general \move{report($\alpha$,
  $Status$)} move which reports on the status of action $\alpha$.
Possible values of $Status$ could be \object{done}, \object{failed},
\object{pending}, \object{initiated} etc.; \move{report($\alpha$,
  \object{done})} would correspond to \move{confirm($\alpha$)}. 




\section{Dialogue Move Engine}




\subsection{Update module}



\subsubsection{Move integration}


\subsubsection{Grounding}



\subsubsection{Question accommodation}


\subsubsection{Plan execution}


\subsubsection{Downdating QUD, ISSUES and ACTIONS}





\subsection{Selection module}



\subsubsection{Action selection}



\subsubsection{Move selection}



\section{Resource interface}



The resource interface can be seen as a mediator between
(domain-independent) modules and (domain-specific) resources.  The
resource interface makes it possible to check and update resources is
update rules. Since the TIS is abstract datastructure containing
objects of various types, it makes sense that resources that are
connected to the TIS should also be regarded as objects of certain
datatypes. 

A straighforward solution to this would be to let each
resource contain a definition of the type of that resource, i.e. a
list of relations, functions, selectors and operations that can be
applied to the resource. However, it turns out that this leads to a
lot of redundancy in the code since each resource object of a certain
type has to define the same type. Instead, the definitions of resource
types reside in the \emph{resource interface} file. The resource
object merely exports a number of predicates in the normal Prolog
way. The resource interface specification uses these predicates to
define the relations etc. that determine how the resource can be
accessed from the modules.






\section{Dialogue plans}

% EXTEND WITH PCs FROM LATER CHs

In this section, we introduce a formalism for representing procedural
plans as sequences of actions. Dialogue plans are implemented in the
domain resource (see Section \ref{sec:domain-resource}).

\subsection{Action sequences and actions}

In the simplest case, a plan consists of a sequence of actions. More
complex plans may also include e.g. conditionals (if-then-else).  In
general, dialogue plans are built from so-called plan constructors.

Action sequences have the form $\langle a_1, \ldots, a_n \rangle$
where $a_i$ \oftype\/ \conttype{Action} ($1\leq i\leq n$). 


All dialogue moves are actions, which means they can be included in
dialogue plans. There are also more abstract actions which however are
connected to dialogue moves. Third, there are actions for manipulating
the information state in various ways. Third, there are actions for
accessing resources, e.g. actions enabling database consultation and
interaction with devices.

In the following, $q$ is a question, $p$ is a proposition, and $a$ is
an action or action sequence.

\subsection{Actions connected to dialogue moves}

In \godis, there are three action types closely related to dialogue
move types.

\paragraph{\action{findout($q$)}}
Find the answer to $q$. This is typically done by asking a question to
the user, i.e., by performing an \move{ask} move, and hoping for an
answer.  The \action{findout} action is not removed until the question
$q$ has been publically resolved. A question is publically resolved
when a resolving answer to the question is in the set of jointly
comitted propositions (i.e. \path{/shared/com})\footnote{Although they
  have not yet been implemented in \godis{}, there are also in
  principle other ways of achieving public resolvedness, e.g. that the
  system finds the information by consulting some resource or by
  inferring from its own private beliefs, and then provides this
  information to the user.  Note that it is necessary that the answer
  is communicated to the user for the findout action to be complete.}
  
\paragraph{\action{raise($q$)}}
  Raise the question $q$. This action is similar to \action{findout},
  except that it is removed from the plan as soon as the
  \move{ask}-move is selected. This means that if the user does not
  answer the question when it has been raised, it will not be raised
  again. This is useful e.g., for requesting optional information.
  
\paragraph{\action{respond($q$)}}
Provide an answer to question $q$. This is done by performing an
\move{answer} move with content $p$, where $p$ is a resolving answer
to $q$.




\subsubsection{Actions for manipulating the information state}



\paragraph{\action{bind($q$)}}

\paragraph{\action{protect($q$)}}

\paragraph{\action{forget($p$)}}

\paragraph{\action{forget\_all}}

\paragraph{\action{forget\_except($p$}}

\paragraph{\action{assume($p$)}}

\paragraph{\action{assume\_issue($q$)}}











\subsection{Resource-related actions}



\paragraph{\action{consultDB($q$)}}


For a typical information-exchange task, the application is a static
database\footnote{Note that ``static'' here does not mean that the
  database cannot be updated. It only means that it is not updated by
  the dialogue system\index{dialgue system}.}.

The \action{consultDB($q$)} action (where $q$ is a question) which
will trigger a database search which takes all the propositions in
\path{/shared/com} and given this looks up the answer to $q$ in the
database. The resulting proposition is stored in \path{/private/bel}.



\paragraph{\action{dev\_do($dev, \alpha$)}}

\paragraph{\action{dev\_set($dev, var, val$)}}

\paragraph{\action{dev\_get($dev, var$)}}

\paragraph{\action{dev\_query($dev, q$)}}

\paragraph{\action{dev\_queryAll($dev, q$)}}


\paragraph{\action{change\_domain}}

\paragraph{\action{change\_language}}




\subsubsection{Conditionals}

\paragraph{\action{if\_then($p$, $a$)}}

If $p$ is in \path{/private/bel} or \path{/shared/com}, then replace
\action{if\_then($p$, $a$)} with $a$; otherwise, delete it.


\paragraph{\action{if\_then\_else($p$, $a_1$, $a_2$)}}
  
If $p$ is in \path{/private/bel} or \path{/shared/com}, then replace
\action{if\_then($p$, $a_1$, $a_2$)} with $a_1$; otherwise, replace it
by $a_2$.



\subsection{Some example plans}

\beg
\begin{tabbing}
\plan{
  \cont{?$x$.price($x$)}}{
  \shortsequence{ \=\\
\>    \action{findout(\cont{?$x.$means\_of\_transport($x$)})},\\
\>    \action{findout(\cont{?$x.$dest\_city($x$)})},\\
\>    \action{findout(\cont{?$x.$depart-city($x$)})},\\
\>    \action{findout(\cont{?$x.$depart-month($x$)})},\\
\>    \action{findout(\cont{?$x.$depart-day($x$)})},\\
\>    \action{findout(\cont{?$x.$class($x$)})},\\
\>    \action{findout(\cont{?return})},\\
\>    \action{if\_then}( \cont{return}, $\langle$ \=     \action{findout(\cont{?$x.$return-month($x$)})},\\
\>\>    \action{findout(\cont{?$x.$return-day($x$)})} $\rangle$ ),\\
\>    \action{consultDB(\cont{?$x.$price($x$)})}\\
}}
\end{tabbing}
\eeg

\beg
\bseg
\item \aplan{\action{vcr\_add\_program}}{
  \sequence{
    \action{findout(\content{?$x$.channel\_to\_store($x$)})}\\
    \action{findout(\content{?$x$.date\_to\_store($x$)})}\\
    \action{findout(\content{?$x$.start\_time\_to\_store($x$)})}\\
    \action{findout(\content{?$x$.stop\_time\_to\_store($x$)})}\\
    \action{dev\_do(vcr, 'AddProgram')}}}{
  \content{done('AddProgram')}}
\eseg
\eeg









\section{Formal semantic representations}

Here we describe the syntax of the simple formal semantic
representation currently used in \godis{}\footnote{It should be noted
  that the \godis{} DME is independent semantic formalism, as long as
  the appropriate semantic properties and relations are defined for an
  application. To implement a GoDiS application with a different kind
  of semantics, the resource interface definitions
  (\texttt{resource\_interfaces.pl}) need to be modified.}. This
description defines a set of content types which are explained and
exemplified below. The symbol ``:'' represents the of-type relation,
i.e., $Expr$ \oftype\/ $Type$ means that $Expr$ is of type $Type$.

\textbf{Atom types}

\conttype{Pred$_n$}, where $n=0$ or $n=1$: n-place predicates, e.g., \cont{dest-city}, \cont{month}\\
\conttype{Ind}: Individual constants, e.g., \cont{paris}, \cont{april}\\
\conttype{Var}: Variables, e.g., $x, y, \ldots, Q, P, \ldots$


\textbf{Sentences}\\

$Expr$ \oftype\/ \conttype{Sentence} iff $Expr$ \oftype\/ \conttype{Proposition} or $Expr$ \oftype\/ \conttype{Question} or $Expr$ \oftype\/  \conttype{ShortAns}


$Expr$ \oftype\/ \conttype{Proposition} if \\
\begin{itemize}
\item $Expr$ \oftype\/ \conttype{Pred$_0$} or
\item $Expr$ = ${pred}_1$(${arg}$), where ${arg}$ \oftype\/ \conttype{Ind} and
  ${pred}_1$ \oftype\/ \conttype{Pred$_1$} or
\item $Expr$ = $\neg P$, where $P$ \oftype\/ \conttype{Proposition} or
\item $Expr$ = \cont{fail($q$)}, where $q$ \oftype\/ \conttype{Question} \\
\end{itemize}

$Expr$ \oftype\/ \conttype{Question} if $Expr$ \oftype\/ \conttype{YNQ} or $Expr$ \oftype\/  \conttype{WHQ} or $Expr$ \oftype\/ \conttype{ALTQ}

?$P$ \oftype\/ \conttype{YNQ} if $P$ \oftype\/ \conttype{Proposition}

\cont{?$x.{pred}_1(x)$} \oftype\/ \conttype{WHQ} if  $x$ \oftype\/ \conttype{Var} and ${pred}_1$ \oftype\/  Pred$_1$

\cont{\{${ynq}_1, \ldots, {ynq}_n$\}} \oftype\/ \conttype{ALTQ} if ${ynq}_i$ \oftype\/ \conttype{YNQ} for all $i$ such that $1\leq i\leq n$

$Expr$ \oftype\/ \conttype{ShortAns} if
\begin{itemize}
\item $Expr$ = \cont{yes} or
\item $Expr$ = \cont{no} or
\item $Expr$ \oftype\/ \conttype{Ind} or
\item $Expr$ = $\neg {arg}$ where $arg$ \oftype\/ \conttype{Ind} 
\end{itemize}





\subsection{Propositions}
\label{sec:isis1-semantics}

Propositions are represented by basic formulae of predicate logic
consisting of an n-ary predicate together with constants representing
its arguments, e.g., \cont{loves(john,mary)}.


In a dialogue system\index{dialgue system} operating in a domain of
limited size, it is often not necessary to keep a full semantic
representation of utterances. For example, a user utterance of ``I
want to go to Paris'' could normally be represented semantically as
e.g., \cont{want(user, go-to(user, paris) )} or \cont{want(u,
  go-to(u,p)) \& city(p) \& name(p, paris) \& user(u)}. \godis{} uses a
\emph{reduced} semantic representation with a coarser,
domain-dependent level of granularity; for example, the above example
will be rendered as \cont{dest-city(paris)}. This reduced
representation is in part a consequence of the use of keyword-spotting
in interpreting utterances, but can arguably also be regarded as a
reflection of the level of semantic granularity inherent in the
underlying domain task. As an example of the latter, in a travel
agency domain there is no point in representing the fact that it is
the user (or customer) rather than the system (or clerk) who is going
to Paris; it is implicitly assumed that this is always the case.

As a consequence of using reduced semantics\index{semantics}, it will
be useful to allow 0-ary predicates, e.g., \cont{return}, meaning
``the user wants a return ticket''. 

The advantage of this semantic representation is that the
specification of domain-specific semantics\index{semantics} becomes
simpler, and that unnecessary ``semantic clutter'' is avoided. On the
other hand, it severely restricts the possibility of providing generic
semantic analyses that can be extended to other domains.

If the database search for an answer to a question $q$ fails the
resulting proposition is \cont{fail($q$)}. We have chosen this
representation because it provides a concise way of encoding a failure
to find an answer to $q$ in the database. 






\subsection{Questions}

Three types of questions are handled by \godis: \textit{y/n}-questions,
\textit{wh}-questions, and alternative questions. Here we describe how these
are represented on a semantic level; the syntactic realization is
defined in the lexicon.

\begin{itemize}
\item \textit{y/n}-questions are propositions preceded by a question mark,
  e.g., \cont{?dest-city(london)} (``Do you want to go to London?'')
\item \textit{wh}-questions are lambda-abstracts of propositions, with the
  lambda replaced by a question mark, e.g., \cont{?$x$.dest-city($x$)}
  (``Where do you want to go?'')
\item alternative questions are sets of \textit{y/n}-questions, e.g.
  \cont{\{?dest-city(london), ?dest-city(paris)\}} (``Do you want to go
  to London or do you want to go to Paris?'') 
\end{itemize}                  

\subsection{Domain ontology (semantic sortal restrictions)}

\godis{} uses a rudimentary ontology consisting of domain-dependent
semantic sortal categories.  Sorts are useful e.g. for distinguishing
non-meaningful propositions from meaningful ones. However, what is
meaningful in one activity may not be meaningful in another, and vice
versa. Therefore, the sortal system is implemented as a part of the
domain knowledge. Another prominent use of sorts is to determine whether
an answer is relevant to (\concept{about}, in Ginzburg\index{Ginzburg,
  Jonathan}'s terminology) a certain question (see Section
\ref{sec:aboutness}).


\subsubsection{Individuals and sorts}


All members of \conttype{Ind} are assigned a sort. For example, the
travel agency domain includes the sorts \cont{city},
\cont{means\_of\_transport}, \cont{class}, etc. The individual
constant \cont{paris} has sort \cont{city} and \cont{flight} has sort
\cont{means\_of\_transport}.






\subsubsection{Sortal hierarchy}


isa


\subsubsection{Sortal correctness of propositions}


The property of a proposition $P$ being sortally correct is
implemented in \godis{} as \concept{sort-restr($P$)}. A proposition is
sortally correct if its arguments fulfil the sortal constraints of the
predicate. For example, the proposition \content{dest\_city($X$)} is
sortally correct if the sort of $X$ is \cont{city}. Sortal constraints
of predicates are implemented in the domain resource, as exemplified
in \nexteg{}.

\beg
sort\_restr( \content{dest\_city($X$)} ) $\leftarrow$ sem\_sort( $X$, \content{city} ).
\eeg










\chapter{Non-DME modules used by GoDiS}



This section describes the modules supplied with TrindiKit that are
used by GoDiS.

The \trindikit\/ package includes a couple of simple modules which can
be used to quickly build prototype systems. 

\begin{itemize}
\item \module{input\_simpletext}: a simple module which reads text
  input from the user and stores it in the TIS
\item \module{output\_simpletext}: a simple text output module
\item \module{intpret\_simple1}: an interpretation module which uses a
  lexicon of key words and phrases to interpret user utterances in
  terms of dialogue moves
\item \module{generate\_simple1}: a generation module which uses a
  lexicon of mainly canned sentences to generate system utterances
  from moves
\end{itemize}





\section{Simple text input module}



The input module  \module{input\_simpletext} reads a
string (until new-line) from the keyboard, preceded by the printing
of an input prompt. The system variable \tisvar{input} is then set to be the
value read.



\section{Text input with simulated recognitions score}



\section{Nuance ASR input}




\section{A simple interpretation module}
\label{sec:module_interpret_simple}

The interpretation module \module{interpret\_simple1}
takes a string of text, turns it into a sequence of words (a
``sentence'') and produces a set of moves.  The ``grammar'' consists
of pairings between lists whose elements are words or semantically
constrained variables.  Semantic constraints are implemented by a set
of semantic categories (\content{location}, \content{month},
 \content{means\_of\_transport} etc.) and synonymy sets.
A synonymy set is a set of words which all are regarded as having the
same meaning.

The simplest kind of lexical entry is one without variables. For
example, the word ``hello'' is assumed to realize a \move{greet}
move.:

\beg
input\_form( [ hello ], \move{greet} )
\eeg

The following rule says that a phrase consisting of the word ``to''
followed by a phrase $S$ constitutes an \move{answer} move with
content \content{to($C$)} provided that the lexical semantics of $S$ is
$C$, and $C$ is a \texttt{location}. The lexical semantics of a word
is implemented by a coupling between a synset and a meaning; the
lexical semantics of $S$ is $C$, provided that $S$ is a member of a
synonymy set of words with the meaning $C$.

\beg
input\_form( [ to$\mid S$ ], \move{answer(to($C$))} $\leftarrow$ lexsem($S$, $C$), location(C).
\eeg

To put it simply, the parser tries to divide the sentence into a
sequence of phrases (found in the lexicon), covering as many words as
possible.



\section{A simple generation module}

The generation module \module{generate\_outputform} takes a
sequence (list) of moves and outputs a string. The generation
grammar/lexicon is a list of pairs of move templates and strings.

\beg
output\_form( \move{greet}, "Welcome to the travel agency!" ).
\eeg

To realize a list of Moves, the generator looks, for each move, in the
lexicon for the corresponding output form (as a string), and then
appends all these strings together. The output strings is appended in
the same order as the moves.










\section{Simple text output module}

The output module \module{output\_simpletext} takes the string in the
system variable \tisvar{output} and prints it on the computer screen,
preceded by the printing of an output prompt. The contents of the
\tisvar{output} variable is then deleted. The module also moves the
contents of the system variable \tisvar{next\_moves} to the system
variable \tisvar{latest\_moves}. Finally it sets the system variable
\tisvar{latest\_speaker} to be the system.




\subsection{Nuance Vocalizer output}


\chapter{The components of a GoDiS application}

This sections outlines the components of a \godis{} application.



\section{File structure}

It is highly advisable to keep trindikit, godis and godis applications
in separate directories. This enables e.g. updating godis to a new
version without touching the applications.

\begin{itemize}
\item trindikit
\item godis/
  \begin{itemize}
  \item godis-basic
  \item godis-grounding
  \item godis-iod
  \item godis-aod
  \end{itemize}
\item godis-apps/
  \begin{itemize}
  \item vcr-godis
  \item ta-godis
  \item $\ldots$
  \end{itemize}
\end{itemize}



\section{The GoDiS application specification file}

This file specifes datatypes, modules and resources used by your
application, as well as additional parameters.

Some applications may have variants, in which case each variant has a
separate specification file (and start file). For example, an application might have one text-based variant and one speech-based variant.

MORE


\subsection{Selecting a GoDiS variant}

Depending on the properties of your application domain, there is a
choice between different variants of GoDiS.

\begin{itemize}
\item godis-basic: multiple simultaneous tasks, information sharing between plans
\item godis-grounding: as godis-basic, plus grounding
\item godis-iod: as godis-grounding, plus accommodation and associated grounding strategies
\item godis-aod: as godis-iod, plus facilities for action-oriented dialogue 
\end{itemize}

\section{The start file}

This file contains specification of directories and files that should
be consulted when running \godis. To load \godis, this file should be
consulted.
 
When building a new application, this is where you specify which
\godis{} version to use, the home directory of you application, the name
of the \godis{} application specification file for your application, and any
additional libraries used by the application.



\section{Resources and the GoDiS resource interface}

The methods available for a resource are not necessarily (or even
usually) defined in the resources itself. Rather, the resource defines
a set of prolog predicates which are utilized by the \textit{resource
  interface}. This interface describes each resource as an object of
some type, and for each type it defines a set of methods (relations,
functions, selectors, and operations). These methods are defined in
terms of the predicates exported from the resource itself.
 

\section{The domain resource}


\subsection{Dialogue plans}



In GoDiS, the domain resource includes a set of dialogue
  plans which contain information about what the system should do in
order to achieve its goals. In dialogue, dialogue plans are loaded
into the information state and executed by update rules, which
``consume'' them step by step. Each plan is associated with a goal
which can either be a question or an action. If the user asks a
question $q$, there is an update rule which looks in the domain
resource for a plan for dealing with $q$, and if one is found, loads
it into the information state.

See Section{sec:dialogue-plans} for an explanation of \godis{}
dialogue plans. 


Dialogue plans are mainly specified using the predicate
\texttt{plan/2}, but there are also some additional predicates.

\begin{itemize}
\item \texttt{plan(Goal, Plan)}
\item \texttt{postcond(Action, Plan)} 
\item \texttt{depends( Q1, Q2)}
\end{itemize}


In the domain resource, sequences of actions are represented by prolog
lists.

The domain resource also specifies a domain-specific ontology which is
used by the dialogue move engine to determine e.g. question-answer
relations (relevance and resolvedness). 



\subsection{Sortal restrictions}

\godis{} uses a rudimentary system of domain-dependent semantic sortal
categories.  For example, the travel agency domain includes the sorts
\cont{city}, \cont{means\_of\_transport}, \cont{class}, etc. All
members of \conttype{Ind} are assigned a sort; for example, the
individual constant \cont{paris} has sort \cont{city} and
\cont{flight} has sort \cont{means\_of\_transport}.

Sorts make it possible to distinguish non-meaningful propositions from
meaningful ones. However, what is meaningful in one activity may not
be meaningful in another, and vice versa. Therefore, the sortal system
is implemented as a part of the domain knowledge.  In \godis{}, the
sorts are mainly used for determining whether an answer is relevant to
(\concept{about}, in Ginzburg\index{Ginzburg, Jonathan}'s terminology)
a certain question (see Section \ref{sec:aboutness}).

The property of a proposition $P$ being sortally correct is
implemented in \godis{}1 as \concept{sort-restr($P$)}. A proposition is
sortally correct if its arguments fulfil the sortal constraints of the
predicate. For example, the proposition \content{dest\_city($X$)} is
sortally correct if the sort of $X$ is \cont{city}. Sortal constraints
of predicates are implemented in the domain resource, as exemplified
in \nexteg{}.

\beg
sort\_restr( \content{dest\_city($X$)} ) $\leftarrow$ sem\_sort( $X$, \content{city} ).
\eeg


\begin{table*}[htbp]
  \centering
  \begin{tabular}{|l|l|}
    \hline
    \textbf{proposition} & \textbf{restriction} \\
    \hline
    \cont{dest\_city($X$)} & $X\in$ \sort{city} \\
    \cont{depart\_city($X$)} & $X\in$ \sort{city} \\
    \cont{how($X$)} & $X\in$ \sort{means\_of\_transport} \\
    \cont{depart\_month($X$)} & $X\in$ \sort{month} \\
    \cont{depart\_day($X$)} & $X\in$ \sort{day} \\
    \cont{class($X$)} & $X\in$ \sort{class} \\
    \cont{price($X$)} & $X\in$ \sort{price} \\
\hline
  \end{tabular}
  \caption{Sortal restrictions on propositions in the Travel Agency domain}
  \label{tab:sortal_restr}
\end{table*}

RELATE TO PROLOG CODE

\subsection{Sortal hierarchy}

The sortal restrictions on proposition is defined in terms of a
hierarchy of semantic sorts, defined by the \texttt{sem\_sort}/2
relation. Since this hierarchy is also useful for the lexicon
resource, it is kept in a separate file (\texttt{semsort\_\ldots.pl}).

As an example, we show the sortal hierarchy from the travel agency domain:


\sortsubcat{top}{
  \sortsubcat{city}{
    \object{paris}\\
    \object{london}\\
    \object{goteborg}\\
    \ldots
    }\\
  \sortsubcat{means\_of\_transport}{
    \object{plane}\\
    \object{boat}\\
    \object{train}
    }\\
  \sortsubcat{month}{
    \object{january}\\
    \object{february}\\
    \ldots
    }\\
  \sortsubcat{day}{
    \object{$1, 2, \ldots, 31$}
    }\\
  \sortsubcat{class}{
    \object{economy}\\
    \object{business}
    }\\
  \sortsubcat{price}{
    \object{\type{Nat}}
    }\\
}

RELATE TO PROLOG CODE

\section{The device/database resource}

\section{The lexicon resource}

\section{The speech recognition grammar resource}


\bibliography{/users/gslt/sl/Latex/sl,/users/gslt/sl/Latex/A-Z,/users/gslt/sl/Latex/trindi,/users/gslt/sl/Latex/SIRIDUS/siridus}

%REFERENCES

%Larsson, Staffan (2002): Issue-based Dialogue Management.




\appendix


\chapter{Installation instructions}



This section describes the steps necessary to download and install
GoDiS and TrindiKit, the toolkit on which GoDiS is built.


\section{Downloading and installing TrindiKit}



TrindiKit can be accessed either via the TrindiKit webpage
(http://www.ling.gu.se/projekt/trindi//trindikit/), via the TrindiKit
SourceForge website (http://sourceforge.net/projects/trindikit/), or
via anonymous CVS from sourceforge. The latter is prefereable if you
want to get the absolutely latest version; however, the documentation
is not always up to date and some parts of the toolkit may be
incomplete. If you want a well-documented release, the first two
options are preferable.



\subsection{Prerequisites}

Trindikit runs under Windows, Unix and Linux. SICStus prolog 3.8 or 
later is needed. In addition, you need the following:

\begin{itemize}
\item JDK 1.4 - needed to compile the java OAA agents and the GUI and to
  run the build script
\item  JRE 1.4 (Included in JDK) - needed to run the build script, see
  INSTALLATION, and to use the java OAA agents and the GUI
\end{itemize}
  
 
In addition the following software is useful but not necessary:
        
\begin{itemize}
\item Nuance ASR 8 - speech recognition software, needed to use the
  module \texttt{input\_nuance\_oaa\_basic.pl}
        
  

\item  Nuance Vocalizer 1 - text-to-speech software, needed to use the module
  \texttt{output\_nuance\_oaa\_basic.pl}
  
 
\item  Open Agent Architecture (OAA) 2.2 (or later) - needed to use the
  TRINDIKIT OAA facilities, and the modules
  \texttt{input\_nuance\_oaa\_basic.pl} and \texttt{output\_nuance\_oaa\_basic.pl}
\end{itemize} 

\subsection{Installing a CVS client}
\label{sec:install-cvs}

\paragraph{Windows}

For Windows, we recommend TortoiseCVS, a graphical CVS client written in Java. I can be found at www.tortoisecvs.org. Download, unzip and install.


\paragraph{UNIX/Linux}


...

Include the following in your .rc.user.d/chs/environment file:

\texttt{setenv CVS\_RSH ssh}



\subsection{Downloading and unzipping TrindiKit}



\subsection{Accessing TrindiKit via anonymous CVS}


\paragraph{Windows}

Right-click on the desktop, and select ``CVS checkout''. A dialogue box
will appear, which should be filled in thus:

\begin{itemize}
\item Protocol: \texttt{:pserver}
\item Server: \texttt{cvs.trindikit.sourceforge.net}
\item Repository folder: \texttt{/cvsroot/trindikit}
\item User name: \texttt{anonymous}
\item Module: \texttt{trindikit}
\end{itemize}

Click ``OK'' and follow the instructions. If you are asked for a
password, just press ``Return'' or click ``OK''.



\paragraph{UNIX/LINUX}


cvs -d :pserver:anonymous@cvs.trindikit.sourceforge.net:/cvsroot/trindikit login  
[supply no password if prompted]
[to checkout our read-only version of trindikit]
cvs -d :pserver:anonymous@cvs.trindikit.sourceforge.net:/cvsroot/trindikit checkout trindikit



\subsection{Installing \trindikit}


\subsubsection{Setting the TRINDIKIT environment variable}



First, you need to set the TRINDIKIT environment variable. 





\paragraph{Windows}

In the Start menu, Go to ``settings''  $\rightarrow$ ``control panel''  $\rightarrow$ ``system'' $\rightarrow$ ``advanced''  $\rightarrow$ ``Environment variables''. Under ``System variables'' click ``New'' and fill in the fields thus:


\begin{itemize}
\item Variable name: \texttt{TRINDIKIT}
\item Variable value: here, enter the path to the ``dist'' directory
  located in your ``trindikit'' directory; the exact path depends on where
  you saved \trindikit{} when downloading it. Example:
  \texttt{C:$\backslash$MyCVS$\backslash$trindikit$\backslash$dist}.
\end{itemize}

\paragraph{UNIX/LINUX}

setenv TRINDIKIT ....


\subsubsection{Running the \trindikit{} build script}


\paragraph{Windows}


Open a DOS window (Start $\rightarrow$ Programs $\rightarrow$
Accessories $\rightarrow$ Command interpreter), go to the
\texttt{trindikit} folder (using the \texttt{cd} command, e.g.
\texttt{cd C:$\backslash$MyCVS$\backslash$trindikit}), and type \texttt{build all}. If
everything is okay, the system should report that the build was
successful.


\subsection{\trindikit{} directory structure (after installation)}

\begin{verbatim}
doc/api - api for the TRINDIKIT java classes
doc/manual - the trindikit manual
lib - external programs and utilities
licenses - various licenses for external programs and utilities
test - for testing 
src/prolog/trindikit - "core" TRINDIKIT
src/prolog/ae - 'agent environment', used for running TRINDIKIT 
   asynchronously
src/prolog/examples - example dialogue systems
src/java - the TRINDIKIT java source tree, including OAA agents and the 
   GUI
dist/classes - the TRINDIKIT java class tree
dist/prolog/trindikit - "core" TRINDIKIT (with appropriate search paths 
   set)
dist/prolog/ae - 'agent environment'
examples - example dialogue systems built using TRINDIKIT
examples/bin - scripts for running example systems
examples/godis - an example system built using TRINDIKIT
\end{verbatim}

\section{Downloading and installing GoDiS}



GoDiS currently has no webpage and is only available via anonymous CVS.



\subsection{Prerequisites}

SICStus Prolog 3.8 or later.


\subsection{Accessing \godis{} via anonymous CVS}

Make sure you have installed a CVS client (see Section
\ref{sec:install-cvs}).

\paragraph{Windows}

Right-click on the desktop, and select ``CVS checkout''. A dialogue box
will appear, which should be filled in thus:

\begin{itemize}
\item Protocol: \texttt{:ext}
\item Server: \texttt{mozart.gslt.hum.gu.se}
\item Repository folder: \texttt{/users/gslt/cvs/repository}
\item User name: \texttt{anonymous}
\item Module: \texttt{godis}
\end{itemize}

Click ``OK'' and follow the instructions.


\paragraph{UNIX/LINUX}

\begin{verbatim}
  [first time you have to login]
  cvs -d :pserver:anonymous@mozart.gslt.hum.gu.se:/users/gslt/cvs/repository login
  [supply no password if prompted]
  [to checkout our read-only version of GoDiS]
  cvs -d :pserver:anonymous@mozart.gslt.hum.gu.se:/users/gslt/cvs/repository checkout godis
\end{verbatim}

\subsection{Installing \godis}


\subsubsection{Setting the GODIS environment variable}

\paragraph{Windows}

In the Start menu, Go to ``settings''  $\rightarrow$ ``control panel''  $\rightarrow$ ``system'' $\rightarrow$ ``advanced''  $\rightarrow$ ``Environment variables''. Under ``System variables'' click ``New'' and fill in the fields thus:


\begin{itemize}
\item Variable name: \texttt{GODIS}
\item Variable value: here, enter the path to the ``dist'' directory
  located in your ``godis'' directory; the exact path depends on where
  you saved GoDiS when downloading it. Example:
  \texttt{C:$\backslash$MyCVS$\backslash$godis$\backslash$dist}.
\end{itemize}


\subsubsection{Running the build script}

\paragraph{Windows}


Open a DOS window (Start $\rightarrow$ Programs $\rightarrow$
Accessories $\rightarrow$ Command interpreter), go to the
\texttt{godis} folder (using the \texttt{cd} command, e.g.
\texttt{cd C:$\backslash$MyCVS$\backslash$godis}), and type \texttt{build all}. If
everything is okay, the system should report that the build was
successful.

\subsection{\godis{} directory structure}





\chapter{Downloading and installing additional software}


\section{Installing Java}


\subsection{Windows}

\textsc{RECOMMEND SOME JAVA DOWNLOAD!}

Make sure that the PATH variable (under Windows) includes a path to
the file \texttt{javac.exe}. If not, search for the file by selecting
Start $\rightarrow$ Search $\rightarrow$ Files and folders. Then,
manually edit the PATH variable as follows: In the Start menu, Go to
``settings'' $\rightarrow$ ``control panel'' $\rightarrow$ ``system''
$\rightarrow$ ``advanced'' $\rightarrow$ ``Environment variables''.
Select ``PATH'' and go to the end; add ``;'' followed by the path to
the \texttt{javac.exe} file.




\section{Downloading and installing OAA}



\section{Installing Nuance ASR}


\subsection{Windows}



\subsection{UNIX/Linux}



\section{Installing Nuance Vocalizer}


\subsection{Windows}



\subsection{UNIX/Linux}




\chapter{Using GoDiS with Nuance v8.0 and Vocalizer}
\label{Using GoDiS with Nuance v8.0 and Vocalizer}



\section{Getting Nuance and Vocalizer to run on your computer}




\subsection{Testing Nuance ASR}



\begin{verbatim}

*k�r ig�ng en license-manager
        (jag tror att jag gjorde en bat-fil som hette  start_nlm el. ngt)

*�ppna ett dosf�nster

*g� till C:\Program\Nuance\v8.0.0\sample-packages o skriv
        'recserver -package digits' (el ngt annat igenk�nningspaket)

*�ppna ett dosf�nster till

*g� till samma st�lle, skriv 'Xapp -package digits lm.Addresses=localhost'
\end{verbatim}



\subsection{Testing Vocalizer}




\section{Configuring your GoDiS application to use Nuance}




\section{Running Nuance and Vocalizer with your application}

Unfortunately, running GoDiS with Nuance and Volcalizer requires going
through a number of steps to get everything running.


\subsection*{Step 1: Compile your nuance grammars}

Go to the directory where the .grammar files are and run the
nuance-compile script on them.

Example:

\texttt{nuance-compile asrg\_travel\_english English.America.1 -auto\_pron
lm.Addresses=localhost}

\subsection*{Step 2: start the Nuance license manager}

This is done by running the \texttt{nlm} script with your license number as argument.


\subsection*{Step 3: Start OAA}


Go to your OAA installation, go to the \texttt{runtime} subdirectory,
and run the \texttt{fac} script.

This requires that OAA is installed, and that the file
\texttt{setup.pl} is directly under the \texttt{C:} directory. This is
achieved by editing the file \texttt{setup1.pl} located in the OAA
runtime directory, changing the text \texttt{ChangeMe} to the name of
your machine.


\subsection*{Edit the setup-nuance script}

This is a script that executes a number of commands automatically. To
make it work, you need to copy the script from an existing
Nuance-connected GoDiS application and change all paths in the file to
match your installations. The file should be stored in the application directory.
\subsection*{Run setup-nuanuce}

Go to your application directory, and type  \texttt{setup-nuance}.

If no error messages have appeared, you can now start your GoDiS application.




\end{document}
