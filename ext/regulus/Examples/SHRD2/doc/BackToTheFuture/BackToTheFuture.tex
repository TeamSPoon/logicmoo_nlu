%
% File acl08.tex
%
% Contact: nasmith@cs.cmu.edu

\documentclass[11pt]{article}
\usepackage{acl08}
\usepackage{times}
\usepackage{url}
\usepackage{latexsym}
\usepackage{psbox}	
\usepackage[latin1]{inputenc}
\setlength\titlebox{8.0cm}    % Expanding the titlebox

\title{Back to the Future: \\Why Spoken Dialogue Systems Need to Rediscover their Roots}

\author{Manny Rayner\\ 
Ecole de Traduction et d'Interpr�tation\\
D�partement de Traitement Informatique Multilingue\\
40, bd. du Pont d'Arve\\
CH-1211 Gen�ve\\
\\
Emmanuel.Rayner@unige.ch}

%\author{~}

\date{}

\begin{document}
\maketitle

\section{Introduction}
\label{Section:Introduction}

\begin{quote}
{\em Reculer pour mieux sauter}\\
Go back to jump better\\
\\
--- French proverb
\end{quote}

When I was an undergraduate, back in the the late 1970s, I came across
a copy of Terry Winograd's {\em Understanding Natural Language}
\cite{Winograd1972} lying on the floor in a friend's room.  I picked it
up and started leafing through it. I was simply amazed. Apparently,
machines could read complex English sentences, understand them at a
deep level, and respond intelligently. I had had no idea that it was
possible at all. I was immediately hooked, and I think that was the
moment when I decided that I didn't want to be a mathematician after
all, but get into AI instead. Wingrad's book, and the SHRDLU system
described in it, inspired me to change my whole career path.

I described this experience recently on a book reviewing site I
sometimes hang out on, and got responses from several people. Someone
asked me what the corresponding thing would be today. What AI system
is going to inspire today's undergraduates the way that SHRDLU
inspired me? I was disturbed to find that I couldn't come up with an
answer. I asked a few friends for suggestions, to see if I was just
blanking. They couldn't come up with anything either.

Back in 1977, I would have predicted that, by now, we would have
speech-enabled systems infinitely more sophisticated than
SHRDLU. Please correct me if you disagree, but I just don't think
that's true. In fact, I don't even think that people are {\em trying}
to build systems like that any more. There was a Call for Papers a few
years ago for a workshop on spoken dialogue systems. In this CFP,
Srinivas Bangalore asked what had happened to work in the area. Did
people decide that the problem had been solved? Did they decide it was
too hard, or that it wasn't actually interesting? It was a good
question.

I've asked people who were around at the time of SHRDLU what they
think happened. Ron Kaplan's explanation seemed plausible to me. The
NLP and spoken dialogue communities became heavily influenced by the
speech recognition community, who made wonderful progress by defining
shared tasks, and steadily chipping away at well-defined objective
functions, in particular Word Error Rate. Unfortunately, I think
history has shown that there was no correspondingly useful objective
function for spoken dialogue systems. DARPA funded two large
initiatives during the 90s and early 00s, with ATIS and COMMUNICATOR;
participants competed on the task of understanding and responding to
questions about airline flight reservations and related topics. There
was, however, no dramatic progress. At the moment, there is no large,
popular speech understanding task.

I worked at NASA for a while, and there's an analogy that has occurred
to me a few times. Back in the 60s, NASA had a clear plan. It was going to
put a man on the Moon, and get him back again alive. It was breathtakingly
audacious, and, despite technology that was very primitive by today's standards,
they succeeded several times. But there was no follow-on plan, and people 
got bored. Funding declined, and manned space-flight started focussing
on the rather less challenging task of maintaining a space station in low Earth orbit.
Unfortunately, no one has ever really understood what the International 
Space Station is for, and there is no genuine enthusiasm for it.
After the Columbia disaster in 2003, the Accident Investigation Board
made all these points, and strongly recommended that NASA needed some
dramatic, motivating goals. The result was the Moon, Mars and Beyond
initiative. First, we would go back to the Moon. Then we would set our
sights on a trip to Mars, and after that to locations further out in the
Solar System.

To me, SHDLU is a bit like Apollo. It was a dramatic demonstration,
and a near-miracle given the currently available technology. It caught
people's imaginations. But now we have become so focussed on simpler,
more readily achievable goals that it's not even obvious any more that
we can do the blocks world very well. Just as with NASA, I'm
suggesting that we should try and get back to the initial ideas. The
first step would be to build a new version of SHRDLU, suitably adapted
to 2009; in particular, it should be speech-enabled, and preferably
also use a real robot rather than a simulated one. Above all, it
should be reasonably robust (the original system was anything
but). Once we've satisfied ourselves that we can do SHRDLU properly, I
suggest defining a new speech understanding task, which is as
ambitious as the SHRDLU task was in 1970. My specific suggestion here
is to aim for a system that can play Bridge, and discuss it. Bridge is
a difficult game, which involves talking about knowledge, uncertainty,
communication, deception and planning. There are strong Bridge engines
available, which are close to the point where they would be able to
handle the underlying reasoning required to support the dialogue; lack
of domain knowledge is usually the biggest problem when building an
ambitious dialogue application. Being able to converse robustly about
Bridge would be a major step forward in spoken dialogue systems.

The rest of this note is organised as follows.
In Section~\ref{Section:SHRD2}, I describe a short-term plan for
reimplementing SHRDLU, which is already in the process of being
implemented. Section~\ref{Section:Bridge} outlines longer-range ideas for building
a system that can play Bridge and talk about it intelligently.
Section~\ref{Section:Conclusions} concludes.

\section{SHRD2}
\label{Section:SHRD2}

We have recently begun work on ``SHRD2'', a speech-enabled
reimplementation of SHDLU. This is being constructed using Regulus
\cite{RegulusBook} a Open Source platform for building speech-enabled
applications. The Regulus strategy is a combination of the rule-based
and data-driven paradigms, and allows grammar-based speech recognisers
to be derived from general resource grammars using semi-automatic
corpus-driven methods. Recognisers can thus quickly be implemented by
linguist developers, using largely reusable resources.  
%A mature
%Regulus resource grammar for English has been available for some time;
%grammars for several other languages have been developed at Geneva
%University under FNRS funding
%\cite{BouillonEA2007TAL,RaynerEAHLT2005}. 
The platform is primarily
designed for tasks which require precise understanding based on deep
analysis, as opposed to the robust surface processing that can be
delivered by machine-learning methods, or the system-initiative,
slot-filling strategies supported by platforms like
VoiceXML\footnote{\url{http://www.voicexml.org/}}. It is thus
particularly suitable for safety-critical domains, and for ``expert''
users who will repeatedly come into contact with the system, and
expect a high level of performance. Regulus has already been used
to build several large applications of this kind. Prominent examples
are NASA's Clarissa procedure
navigator\footnote{\url{http://ic.arc.nasa.gov/projects/clarissa/}}
(tested on the International Space Station in 2005), Geneva
University's MedSLT medical speech translator
\cite{BouillonEA2008AMTA}, and Ford Research's
SDS in-car information system, which was voted first at the 2007 Ford
Internal Technology Fair.

\begin{figure*}
Person: PICK UP A BIG RED BLOCK.\\
Computer: OK. (does it) \\
Person: GRASP THE PYRAMID. \\
Computer: I DON'T UNDERSTAND WHICH PYRAMID YOU MEAN. \\
Person: FIND A BLOCK WHICH IS TALLER THAN THE ONE YOU ARE HOLDING AND PUT IT INTO THE BOX. \\
Computer: BY "IT", I ASSUME YOU MEAN THE BLOCK WHICH IS TALLER THAN THE ONE I AM HOLDING. \\
Computer: OK. (does it) \\
Person: WHAT DOES THE BOX CONTAIN? \\
Computer: THE BLUE PYRAMID AND THE BLUE BLOCK. \\
Person: WHAT IS THE PYRAMID SUPPORTED BY? \\
Computer: THE BOX. \\
Person: HOW MANY BLOCKS ARE NOT IN THE BOX? \\
Computer: FOUR OF THEM. \\
Person: IS AT LEAST ONE OF THEM NARROWER THAN THE ONE WHICH I TOLD YOU TO PICK UP? \\
Computer: YES, THE RED CUBE. \\
Person: IS IT SUPPORTED? \\
Computer: YES, BY THE TABLE. 
\caption{Beginning of the classic SHDLU dialogue.}
\label{Figure:SHDLUDialogue}
\end{figure*}

When the original SHDLU system was written, there was no established
concept of systematic testing using a corpus, with clean separation of
training and test data. The closest thing to a corpus is the famous
SHRDL demo dialogue, posted for example at \url{http://hci.stanford.edu/~winograd/shrdlu/}.
The beginning of the dialogue is shown in Figure~\ref{Figure:SHDLUDialogue}.
Using the demo dialogue as a starting point, we are in the process of collecting
a proper corpus, which right now contains about 200 utterances.
Using this corpus, we have implemented initial versions of lexicon,
semantics and other modules. After a couple of weeks of work, about
80\% of the corpus parses, and about 50\% produces a scoped
logical form. Figure~\ref{Figure:SHRD2Example} shows different
processing stages for a simple example. The grammar can be compiled
into a Nuance recognizer. Anecdotally, recognition is reasonable.
The main problems, as we expected, are in confusion of function words;
in particular, ``in'' is often confused with ``on'', and ``a'' with ``the''. 
So, for example, ``Put the cube in the box'' might easily be misrecognized
as ``Put {\bf a} cube {\bf on} a box''.

\begin{figure*}

{\bf Surface form:}\\  
``pick up a big red block''
~\\
~\\  
{\bf Representation produced by parser/recogniser:} \\
\begin{verbatim}
[[imp, 
  form(imperative, 
       [[pick_up,
         term(pro,you,[]),
         term(a,block,[[size,big],[color,red]]),
         up]])]]
\end{verbatim}
~\\
{\bf Scoped semantic representation:}\\
\begin{verbatim}
[imp, 
 quant(def_sing, A, 
       [[you,A]], 
       quant(exist, B, 
             [[block,B], [size,B,big],
              [color,B,red]],  
             imperative(quant(exist,C,[[pick_up,C,A,B]],
                        true))))]] 
\end{verbatim}

\caption{SHRD2 processing example, showing different processing stages.}
\label{Figure:SHRD2Example}
\end{figure*}

Our short-term plan is to address these and other problems, make coverage more
robust, and attach a back-end system which will actually respond to questions
and commends. Initially, the back-end will consist of a simulated version of the
blocks-world, as in the original SHRDLU. We will probably also start experimenting
with the idea of attaching a real robot arm. This work may be carried out by
students at Santa Clara University.
%, or by researchers at the new robotics
%group at PARC, Palo Alto.

\subsection{Using SHRD2 as a shared resource}

Looking further ahead, one possible medium-term plan for SHRD2 that
has been suggested to us is to develop it into a shared resource that
can be used by other groups interested in doing research on spoken dialogue
understanding.  The approach we have in mind is to place a version of
the system on the web, and include hooks and APIs which make it possible
to replace components with third-party versions. Thus, for example,
someone who wanted to use a different natural language processing
engine could upload an executable which received an N-best list in XML
format, and produced a semantic representation.  Alternately, they
could do the same thing offline: run the current set of speech
examples to produce a file of N-best lists, download that file,
process it locally to produce a file of semantic representations,
upload that file back to the server, and then run another offline
script to process the semantic representations through response
generation.

The technically challenging aspect of this exercise is the question of
how to represent discourse and task context, since processing will generally
be context-dependent. For example, pronouns like ``it'' or ``them'' will
often refer back to the previous discourse, while definite descriptions like
``the blue cube'' or ``the largest block'' will refer to objects in the 
environment. Processing modules will in general need to input and/or
output a context object, which encodes this information. The question is
whether the context can be represented in a sufficiently theory-neutral 
way that a reasonable cross-section of groups would be able to use
together with their own frameworks. We will start by trying to make the
context objects as uncontroversial as we can, and then discuss issues with
other potentially interested groups.

\section{Bridge}
\label{Section:Bridge}

SHDRLU was extremely ambitious for 1969, and at the time it was
inspirational to see even a partial success on such a challenging
task. We hope that SHRD2 will show that real progress has happened in
the intervening 40 years, and that we can now build a fairly robust
speech-enabled system for the block-world domain. We then want to move
to a new domain, which will be as ambitious for 2009 as SHDLU was for
1969. We suggest that Bridge would be a suitable candidate.

Bridge is a complex card game played between two teams, in which
players in general need to cooperate with their partner and attempt to
confuse their opponents. It constantly involves reasoning about
knowledge, belief, probabilities and plans.  For example, during the
bidding phase it is often important to think about what one's partner
will know as a result of a bid, and to what extent this will help them
make useful decisions. During the following play of the cards, it is
common to reason about what the opponents are likely to have inferred
from the bidding and the play to date.

In contrast to games of perfect information, like chess, it is by no
means obvious even that a very strong player will choose the
alternative that is objectively correct. To give a example, suppose
that one opponent (East) has the Ace of Spades, their partner (West)
has information leading them to believe that East has one Ace, and
West's leading a Spade would be enough to defeat the
contract. However, if there is no reason for West to think that East's
Ace is more likely to be in Spades than in Hearts, it is rational to
expect them to make the correct decision at most 50\% of the time.  It
is common to look ahead to future situations like this, and
reason about how useful information will be to the opponents or to
one's partner.

The language required to talk about issues like these implies a
qualitative leap beyond that used in domains like the
blocks-world. One would be required to give semantics to words and
constructions like ``believe X'', ``be able to guess whether X'',
``prefer X to do Y'', ``trick X into believing Y'', and so on.
Linguistics and philosophers working in formal semantics have been
extremely interested in modal and intension constructions of this kind
since at least the time of Frege, and during the 70s and 80s this work
used to be referenced with some frequency in the AI literature as
well.  The impoverished nature of the domains currently used in
practical spoken dialogue system implementation has meant, however,
that what contact there was between the two research areas has almost
disappeared.

The reason why dialogue system builders have stopped looking at these
difficult constructions is, of course, that it is in most domains very
hard to reason about them quantitatively. Reasoning of this kind is
not trivial in Bridge either, but there are at least well-understood
approaches which give good purchase on the key concepts. ``Knowledge''
and ``belief'' can be modelled by performing random generation of card
distributions which match constraints derived from previous bidding
and play, an idea which has been explored in the GIB system
\cite{Ginsberg99}. As pointed out in that paper, the approach
implemented in GIB was still fairly simplistic, and (at least at the
time of writing) was not capable of ``information gathering'' plays.
It should be possible to address these shortcomings, though it may be
computationally expensive to do so. 

The basic reason for feeling optimistic is that ability to generate
distributions essentially supplies a possible-worlds semantics; this
means that we have access to the large literature on possible-worlds
semantics for modal operators, and can use it to formalize concepts
relating to actions and uncertain beliefs. We are not expecting this
to be straightforward, but there are clear ways to make progress. The
point, as with SHRDLU, is that attempting to solve these problems
would tie together a considerable body of currently unrelated
research, within the framework of a single intuitively meaningful
domain. If we can get to the point of having a working, implemented,
generally accessible framework and a substantial corpus of examples,
then history suggests that progress will happen more or less of
itself, as different groups compete on the task of improving their
scores on the corpus. 

\section{Summary}
\label{Section:Conclusions}

To summarize: when one compares the original goals of dialogue system
research, as exemplified in SHRDLU, with currently implemented
systems, it seems clear that the early sense of excitement and focus
has been lost.  It is less clear why this is. What I have suggested
here is a two-part plan. First, we should reimplement a robust,
speech-enabled version of SHRDLU.  This would demonstrate to
everyone's satisfaction that there has been genuine progress since
1970, and that the goals set then were merely ambitious, rather than
fraudulent or unrealistic. When this has been achieved, I then suggest
attempting a new domain, which should be as ambitious now as SHDLU was
in its own timeframe. Leveraging the major advances that have been
made in games systems, I suggest that Bridge could be a suitable
domain. 

The reason why AI has always been interested in games is that a game
defines a closed microworld. In some games, like Chess, machines now
play at world championship level, and human experts take their advice
very seriously. A machine that could both play a game well and discuss
it credibly would fulfill all the criteria necessary to be accepted on
equal terms as a member of the society of people who played that
game. If we could create such a system, I at least would feel that AI
had passed an important milestone, and I would find it very motivating
to work towards that goal. 

\section*{Acknowledgements}

I've discussed the ideas outlined here with many people in the AI,
computational linguistics, spoken dialogue systems and Bridge
communities. I'd particularly like to thank Srinivas Bangalore, Cathy
Chua, Dick Crouch, Beth Ann Hockey, Ed Hovy, Ron Kaplan, Leon
Stirling, Andrew Webb and Annie Zaenen. I'd be delighted to get further
comments and criticism, and will send out new drafts as appropriate.

\bibliographystyle{acl}
\bibliography{ACL2009}

\end{document}